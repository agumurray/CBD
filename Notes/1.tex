\section{Conceptos generales de bases de datos}
\subsection{Conceptos Basicos}
\begin{itemize}
\item Se considera \textbf{Base de Datos}, a una coleccion o conjunto de datos interrelacionados con un proposito especifico vinculado a la resolucion de un problema del mundo real.
\item Cualquier informacion dispuesta de manera adecuada para su tratamiento por una computadora.
\item Una coleccion de archivos diseñados para servir a multiples aplicaciones.
\end{itemize}

\subsection{Origenes de las BD}

En los origenes de las \textbf{BD}, las unidades de almacenamiento de gran volumen eran muy lentas, entonces se buscaba reducir el acceso a los mismos. Para esto, era necesario repetir datos en distintos puestos de trabajo y al final del dia actualizar los mismos en todas las unidades para evitar problemas de perdida/modificacion de informacion. \\
Con el tiempo, la tecnologia fue avanzando y los sistemas de informacion evolucionaron. Se logra integrar aplicaciones, interrelacionar archivos y eliminar la redundancia de datos.

\subsection{Gestores de Bases de Datos}
Un \textbf{Sistema de Gestion de Bases de Datos (SGBD)} consiste en un conjunto de programas necesarios para acceder ya administrar una BD. \\
Actualmente, cualquier sistema de software necesita interactuar con informacion almacenada en una BD y para ello requiere del soporte de un SGBD. \\
Un SGBD posee dos tipos diferentes de lenguajes: uno para especificar el esquema de una BD, y el otro para la manipulacion de los datos.\\
La definicion del esquema de una BD implica:
\begin{itemize}
  \item Diseño de la estructura que tendra efectivamente la BD.
  \item Describir los datos, la semantica asociada y las restricciones de consistencia.
\end{itemize}

Para ello se utiliza un lenguaje especial, llamado \textbf{Lenguaje de Definicion de datos (LDD)}.El resultado de compilar lo escrito con el LDD es un archivo llamado Diccionario de Datos.Un Diccionario de Datos es un archivo con metadatos, es decir, datos acerca de los datos.

Los objetivos mas relevantes de un SGBD son:
\begin{itemize}
  \item \textbf{controlar la concurrencia: }varios usuarios pueden acceder a la misma informacion en un mismo periodo de tiempo. Si el acceso es para consulta, no hay inconvenientes, pero si mas de un usuario quiere actualizar el mismo dato a la vez, se puede llegar a un estado de inconsistencia que, con la supervicion del SGBD, se puede evitar.
  \item \textbf{Tener control centralizado: }tanto de los datos como de los programa que acceden a los datos.
  \item \textbf{Facilitar el acceso a los datos: }dado que provee un lenguaje de consulta para recuperacion rapida de informacion.
  \item \textbf{Proveer seguridad para imponer restricciones de acceso: }se debe difinir explicitamente quienes son los usuarios autorizados a acceder a la BD.
  \item \textbf{Mantener la integridad de los datos: }esto implica que los datos incluidos en la BD respeten las condiciones establecidas al definir la estructura de la BD y que, ante una falla del sistema, se posea la capacidad de restauracion a la situacion previa.
\end{itemize}
